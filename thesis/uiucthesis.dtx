\def\fileversion{v1.0n} \def\filedate{1997/09/15}
% \iffalse	% this is a METACOMMENT !
%% Package "uiucthesis" for use with LaTeX2e.
%
%<package>\NeedsTeXFormat{LaTeX2e}
%<package>\ProvidesPackage{uiucthesis}
%<package>         [\filedate\ \fileversion\ UIUC Thesis (DLH)]
%
%<*driver>
\documentclass{ltxdoc}
\begin{document}
\title{The \textsf{uiucthesis} package
       \thanks{This file has version number \fileversion,
       last revised \filedate.}}
\author{David Hull\\hull@cs.uiuc.edu}
\date{\filedate}
\maketitle
\DocInput{uiucthesis.dtx}
\end{document}
%</driver>
% \fi
%
% \CheckSum{0}
%
% \MakeShortVerb{\|}
%
% \def\pkg#1{\textsf{#1}}
% \def\env#1{\textsf{#1}}
% 
% \iffalse
%<*example>
% \fi
% \begin{abstract}
% Load the \pkg{uiucthesis} package with the standard \LaTeX2e book
% class to produce a document that conforms
% to the format described in
% \emph{Instructions for Preparation of Theses}\cite{Instructions94}.
% \end{abstract}
%
% \section{The User Interface}
%
% This section describes how to use the \pkg{uiucthesis} package to
% produce a thesis satisfying the format requirements of the
% Grad College at UIUC.
% I assume that you are familiar with \LaTeX, and highly
% recommend that anyone attempting to use \LaTeX\ to produce a
% thesis have access to a copy of the \LaTeX\ book\cite{Lamport86}.
%
% \subsection{Using \pkg{uiucthesis}}
% 
% To write a thesis, you load the UIUC thesis definitions
% by loading the \pkg{uiucthesis} package at the beginning of
% your \LaTeX\ document, immediately after the |\documentclass| command.
% For example,
% \begin{quote} \hrule \begin{verbatim}
\documentclass[11pt,oneside]{book}
\usepackage[draftthesis,fullpage]{uiucthesis}
% \end{verbatim} \hrule \end{quote}
% \iffalse
%<example>
% \fi
%
% \DescribeMacro{[draftthesis]}
% \DescribeMacro{[fancy]}
% \DescribeMacro{[fullpage]}
% The \pkg{uiucthesis} package provides three options.
% The |[draftthesis]| option causes each page to have a header
% proclaiming the document to be a draft copy, along with the
% current time and date.
% It also omits the preliminary part of the
% thesis---currently only the copyright page.  
% This is marginally useful for saving paper before you
% produce the final version of the thesis.
% The |[fancy]| style option produces slightly fancier chapter headings.
% The |[fullpage]| style option makes the margins as small as the
% format requirements allow.  Because wide text columns are ugly
% this is not the default, but is provided as an option because people
% seem to want it.
% The |[fancy]| and |[fullpage]| options are incompatible---choose
% one or the other.
%
% \DescribeMacro{\draftheader}
% If you don't like the header that the the |[draftthesis]| option
% creates, you can redefine the |\draftheader| command so that it
% produces whatever text you want.
%
% \DescribeMacro{\thesisspacing}
% The \pkg{uiucthesis} package loads \pkg{setspace} for the
% line spacing commands.  See the documentation in that package
% for more information on the commands it provides.
% By default, \pkg{uiucthesis} uses one and a half line spacing,
% or double spacing if the |[fullpage]| option is specified.
% If you're unhappy with that, you can override it by redefining the
% |\thesisspacing| command.
%
% \subsection{The Title Page}
% 
% The |\maketitle| command is redefined so that it creates a
% title page with the correct format for a thesis at UIUC.
%
% \DescribeMacro{\phdthesis}
% \DescribeMacro{\msthesis}
% \DescribeMacro{\department}
% \DescribeMacro{\college}
% Use the |\phdthesis| or |\msthesis| to set the correct thesis type.
% Set your department with |\department{|\meta{department}|}|.
% This defines the field your degree will be in, so leave out
% ``Department of.''
% The default department is ``Computer Science.''
% Define your college with |\college{|\meta{college}|}|.
% The default is college is ``Graduate College;''
% you shouldn't need to change it.
%
% \DescribeMacro{\schools}
% Use |\schools{|\meta{school list}|}| to list the previous degrees
% you have received and the schools that you received them from.
% Separate multiple degrees with |\\|.
%
% \DescribeMacro{\degreeyear}
% Use |\degreeyear{|\meta{year}|}| to define the year in which
% you will receive your degree.  The default is the current year.
%
% Here's how to produce the example on Page~25 of \cite{Instructions94}.
% \begin{quote} \hrule \begin{verbatim}
\begin{document}

\title{Coffee Consumption of Graduate Students \\
       Trying to Finish Dissertations}
\author{Juan Valdez}
\department{Food Science}
\schools{B.A., University of Columbia, 1981\\
         A.M., University of Illinois, 1986}
\phdthesis
\degreeyear{1994}
\maketitle
% \end{verbatim} \hrule \end{quote}
% \iffalse
%<example>
% \fi
%
% \subsection{Front Matter}
%
% \DescribeMacro{\frontmatter}
% Typically, a thesis might have an Abstract, a Dedication, some
% Acknowledgments, and a Preface before the Table of Contents.
% Use the |\frontmatter| command to start this preliminary section
% of the thesis.
% The |\frontmatter| command sets the page number of the next page
% to roman numeral iii.  (The title page is page i, and the certificate
% of committee approval, the ``red-bordered form,'' is page ii.)
%
% Since none of the sections in the front part of the thesis
% should appear in the Table of Contents, you should use introduce
% them with the |\chapter*| command.
%
% After this front matter comes the Table of Contents,
% List of Tables, List of Figures, etc.  Use the standard \LaTeX\ 
% commands |\tableofcontents|, |\listoftables|, |\listoffigures|, etc.,
% to generate them.
% In the \pkg{uiucthesis} format these lists are all single spaced.
%
% Here's an example of what the front matter of a typical
% thesis looks like.  First comes the Dedication, which is optional.
% The heading on a Dedication is not required, but it must have
% a page number.  In this example, the dedication is centered vertically
% on the page.
% \begin{quote} \hrule \begin{verbatim}
\frontmatter

%% Create a dedication with no heading, centered vertically
%% on the page.
\newpage
\leavevmode\vfill
\begin{center}
To me.
\end{center}
\vfill

\chapter*{Acknowledgments}

This thesis would not have been possible without the
the help of the many graduate students who bravely
consumed many gallons of coffee, thus providing
the abundance of raw data for my research.

%% The thesis format requires these lists to come in the following
%% order:
%%
%% Table of Contents
%% List of Tables
%% List of Figures or List of Illustrations
%% List of Symbols and/or Abbreviations

\tableofcontents
\listoftables
\listoffigures
% \end{verbatim} \hrule \end{quote}
% \iffalse
%<example>
% \fi
%
% If you want a List of Symbols or Abbreviations you have to create
% it yourself, but it isn't difficult.  For example to create a List
% of Abbreviations you could use:
% \begin{quote} \hrule \begin{verbatim}
%% Create a List of Abbreviations.

\chapter*{List of Abbreviations}
\begin{description}
\item[CA] Caffeine Addict.
\item[CD] Coffee Drinker.
\end{description}
% \end{verbatim} \hrule \end{quote}
% \iffalse
%<example>
% \fi
%
% \subsection{Main Matter}
%
% \DescribeMacro{\mainmatter}
% Begin the main body of your thesis with the |\mainmatter| command.
% It resets the page number to arabic numeral 1.
% You can now use any of the commands defined by the
% the book document class to write your thesis.  
%
% In the following example, each of the chapters
% has been broken out into separate files that are inserted into
% this main file with the |\include| command.  This allows the
% thesis to be proofed quickly while it is being revised with
% the |\includeonly| command.
%
% \begin{quote} \hrule \begin{verbatim}
\mainmatter

\chapter{Introduction}
\label{ch:introduction}

The relatively new field of quantum computing has seen rapid growth in
the past two decades.  Quantum computing spans the theoretical and
applied sides of both computer science and quantum physics.  From its
beginnings as a thought experiment, its growth into formal system, and
finally its detailed analysis and construction, the development of
quantum computing has paralleled the early development of classical
computing.

In Section \ref{sec:mot} we briefly examine some of the reasons for
interest in quantum computing.  We then turn our attention to early
results and prominent quantum algorithms in Section \ref{sec:early}.
We review the theory and notation of quantum computing in Section
\ref{sec:quant}.  Following that we underline the importance of lower
bounds in Section \ref{sec:lower}, and summarize our results in
Section \ref{sec:sumofres}.

\section{Motivation}
\label{sec:mot}

\subsubsection{Possible Violation of the Polynomial Church-Turing Thesis}

One of the fundamental axioms of computer science is the Church-Turing
thesis, which states that any function computable by a ``realistic''
computer can be computed by a Turing machine.  An extension of the
Church-Turing thesis, the Polynomial Church-Turing thesis states:
\begin{quote}
\ldots any reasonable attempt to model mathematically
computer algorithms and their time performance is bound to end up with
a model of computation and associated time cost that is equivalent to
Turing machines within a polynomial \cite{papa94complexity}.
\end{quote} 
This says essentially that all physically realizable computing devices
are, within a polynomial factor, equivalent to one another in their
time complexity.

In the early 1980's physicist Richard Feynman observed that no
classical computer can simulate a quantum mechanical system of
particles without incurring exponential slowdown
\cite{williams98quantum}.  He also suggested that a computer that
behaves in a quantum-mechanical way could potentially simulate such
systems without exponential slowdown \cite{williams98quantum}.  The
possibility that a quantum computer could violate the polynomial
Church-Turning thesis made the study of quantum computation appealing,
and provided a strong incentive for studying quantum time complexity.
Many of the earliest problems and algorithms for quantum computers
were explicitly designed to show tasks that a quantum computer
performs exponentially faster than a classical Turing machine.
	
\subsubsection{Hardware Trends Toward Quantum Sizes}

Another strong incentive to study quantum computing is the
miniaturization of classical computing components.  The size of
transistors and memory elements has shrunk at an exponential rate.  At
the current rate, sometime around 2020 the number of atoms used to
represent a bit will be 1 \cite{williams98quantum}.  At this scale,
quantum mechanics will dominate the behavior of the memory element.
Whether or not quantum computing will provide a launch pad for a new
generation of computing devices is irrelevant to the practical need to
understand and manipulate systems so small that their behavior is
dictated by quantum physics.

\section{Early Results}
\label{sec:early}

\paragraph{Deutsch's Universal Quantum Computer:}
In 1985 David Deutsch published his seminal paper
\emph{Quantum theory, the Church-Turing principle and the universal
quantum computer} \cite{deutsch85quantum}.  In this paper Deutsch
defined a quantum generalization of the classical Turing machine,
showed that all Turing computable functions are also computable by his
universal quantum computer, and exhibited a task that his universal
quantum computer is more efficient for than any classical restriction
of it \cite{deutsch85quantum}.

Following this paper, researchers identified several toy problems that
a quantum computer is exponentially faster for than a classical Turing
machine \cite{brassard97exact} \cite{berthiaume92oracle}.
Unfortunately, these were all contrived problems with no practical
application.  While the potential for exponential speedup fostered
great curiosity, the study of quantum computing remained primarily
academic.

\subsubsection{Prominent Quantum Algorithms}

\paragraph{Shor's Algorithm:}
The status of quantum computing as a matter of academic curiosity
changed rapidly in 1994 when Peter Shor published his paper
\emph{Algorithms for Quantum Computation: Discrete Logarithms and
Factoring}.  The primary result of the paper was a polynomial time
quantum algorithm for factoring large integers
\cite{shor94algorithms}. It is not known if there is a classical
algorithm for factoring large integers efficiently, but the best
algorithms published thus far are exponential
\cite{williams98quantum}.  The presumed difficulty of factoring 
large integers is the basis for most modern cryptography.  The
importance of cryptography and its potential frailty in the face of
Shor's algorithm argue for earnestly researching the practicality of
constructing a quantum computer; this endeavor is currently ongoing in
multiple corporate, government, and academic research facilities
\cite{ibm01shor} \cite{chuang98quantum}.

\paragraph{Grover's Algorithm:}
In 1996 L. K. Grover published his paper \emph{A Fast Quantum
Mechanical Algorithm for Database Search} providing a $O(\sqrt{N})$
time algorithm for finding a single marked element in an unsorted
database of $N$ elements \cite{grover96fast}; the best possible
classical algorithm requires $\Omega(N)$ time.  This unordered search
problem was not the first problem for which a quantum algorithm was
shown to be asymptotically faster than any classical algorithm, but it
was the first such problem of real utility.  Grover's algorithm is
optimal to within a constant factor, so the potential speedup of any
quantum algorithm for the unordered search problem is moderate: a
quadratic factor.  While Shor's algorithm may be of more immediate
utility, Grover's algorithm seems more interesting in a theoretical
sense, as it highlights an area of fundamental superiority in quantum
computation.

\section{Quantum Computing}
\label{sec:quant}

The study of quantum computing is frequently opaqued by the demands
that it places on the investigator's sophistication in quantum
mechanics.  We try in this section to present the basics of quantum
computation and some common notation without delving deeply into
quantum physics.  For more detailed information the author suggests
\emph{Quantum Computation} by Andr\'{e} Berthiaume
\cite{berthiaume97quantum}.

\subsubsection{Bits and Qubits}

The most fundamental building block of the classical computer is the
bit.  A bit is a variable with only two possible values: 0 or 1.  The
smallest conceivable storage for a bit involves a single elementary
particle of some sort.  Consider a particle with a spin-1/2
characteristic that when measured is either $+1/2$ or $-1/2$.  We
could encode 1 to be $+1/2$ and 0 to be $-1/2$, and if we could
measure and manipulate the spin of such a particle, then we could
theoretically use this particle to store one bit of information.  The
spin-1/2 particle, or any two state system that behaves in a quantum
manner, could instead be the fundamental building block of a quantum
computer.  We will call a two state quantum system a \emph{qubit}, to
denote that it is analogous to a classical bit.  When a classical bit
is measured, the value observed will always be the value stored.
Quantum physics states that when we measure a qubit we will find it in
one of two states, which we can label 0 and 1.  The differences
between the qubit and the bit arise from the possible state of a qubit
between measurements.

\subsubsection{State Vectors and Dirac Notation}

To quantify the state of our qubit when it is not being measured, we
introduce the concept of the \emph{state vector}, which will
completely describe the state of our qubit, and later the state of a
quantum register which we construct from multiple qubits.

\paragraph{State Vectors:} 
We can describe the state of any quantum system by a state vector in a
Hilbert space.  A Hilbert space is a complex linear vector space
$\Complex^{N}$ \cite{williams98quantum}.  In the Hilbert space for a
state vector describing an $N$-state quantum system there will be $N$
perpendicular axes, which correspond to the measurable states of the
system.  These are called basis states or \emph{eigenstates}.  In
general, the total state of a quantum system can be any complex linear
combination of the basis states.  The Hilbert space for a single qubit
has two perpendicular axes, one corresponding to the 0 state, and the
other corresponding to the 1 state.

\paragraph{Dirac Notation:}

The standard notation for a state vector is a \emph{ket vector}
$|\Psi\rangle$.  For example, a vector in $\Real^{3}$ with basis
$\hat{i},\hat{j},\hat{k}$ is typically written as
\[
\vec{v} =  a\hat{i}  +  b\hat{j}  +  c\hat{k}  =  (a,b,c)^{T} .
\]
In Dirac notation, it would be written as
\[
|v\rangle  =  a|i\rangle  +  b|j\rangle +  c|k\rangle  = 
(a,b,c)^{T}. 
\]
The term ket and this notation come from the physicist Paul Dirac who
wanted a concise way of writing formulas involving row and column
vectors.  He referred to row vectors as \emph{bra vectors} represented
as $\langle y|$.  The inner product of a \emph{bra} and a \emph{ket}
vector is written $\langle y|x\rangle$, and is called a \emph{bracket}
\cite{williams98quantum}.

\subsubsection{Superposition}

The projection of the state vector onto one of the axes of its Hilbert
space shows the contribution of that axis's eigenstate to the whole
state.  A classical bit's state vector can only lie along one of the
two axes.  The state of a qubit can be any vector $|X\rangle$ in the
Hilbert space with $\langle X|X\rangle = 1$; such a state vector is
called \emph{normalized}.  The inner product of a vector $|X\rangle =
(x_{0},x_{1},\ldots, x_{N-1})^{T}$ with itself is $|x_{0}|^{2} +
|x_{1}|^{2} + \ldots + |x_{N-1}|^{2}$, where $|a+ib|^{2} = a^{2} +
b^{2}$.  More generally $\left\langle X|Y\right\rangle = \sum_{i =
0}^{N-1} x_{i}^{*}y_{i}$, where $(a + ib)^{*}$ is $a - ib$.

Let $x_{1}$ be the eigenstate corresponding to the 1 state, and let
$x_{0}$ be the eigenstate corresponding to the 0 state.  We can write
any state $|X\rangle$ as $w_{0}|x_{0}\rangle + w_{1}|x_{1}\rangle$,
where $w_{0}, w_{1}$ are the complex projections of $|X\rangle$ onto
the eigenstates such that $|w_{0}|^{2} + |w_{1}|^{2} = 1$.  When the
qubit with state vector $X$ is measured, we are guaranteed to find it
to be in either the state $1 |x_{0} \rangle + 0 |x_{1} \rangle =
|x_{0}\rangle$ or the state $0 |x_{0} \rangle + 1 |x_{1} \rangle =
|x_{1}\rangle$.

More generally, the Hilbert space of an $N$-state quantum system is
$\Complex^{N}$.  As with the two state system, when we measure our
$N$-state quantum system we will always find it to be in exactly one
of the eigenstates.  The system is allowed to exist in any complex
linear superposition of the $N$ states between measurements.  An
$N$-state quantum system with eigenstates $x_{0}, x_{1},\ldots,
x_{N-1}$ can be fully described by the vector
\[|X \rangle = \sum_{k = 0}^{N-1} w_{k} |x_{k} \rangle, 
\textrm{ where }
\sum_{k = 0}^{N-1} |w_{k}|^{2} = 1.
\]

Our state vector can exist in a linear superposition of eigenstates,
but we can only measure the state vector to be in one of the
eigenstates.  When the state vector is observed, it makes a sudden
discontinuous jump to one of the eigenstates.  When measurement is
performed the state vector is said to collapse
\cite{williams98quantum}.  For an $N$-state quantum system with a 
normalized state vector, the probability that the state vector will
collapse into the $j$th eigenstate is simply $|w_{j}|^{2}$.  The
coefficient $w_{j}$ is called the \emph{amplitude} of eigenstate
$|x_{j}\rangle$.

We can construct a quantum memory register out of the qubits described
in the previous section.  Just as in a classical computer, a quantum
computer will perform calculations by manipulating its memory register
from some start state to some final state.  Note that a quantum
register composed of $N$ qubits requires $2^{N}$ complex numbers to
completely describe its state vector, as an $N$-qubit register has
$2^{N}$ basis states.

\subsubsection{Unitary Operators}

Not all functions on a quantum memory register preserve the
superposition of the state vector.  For example, measurement destroys
the superposition in the register.  Operations that collapse the state
vector are called \emph{measurements}, and any complex linear
transformation of the state vector is called an
\emph{operator}.  We can represent any operator on an $N$-bit 
quantum memory register in $\Complex^{2^{N}}$ as a matrix
\[T = \left( \begin{array}{cccc}
	T_{00} & T_{01} & \ldots & T_{0(2^{N}-1)} \\ T_{10} & T_{11} &
	\ldots & T_{2(2^{N}-1)} \\ \vdots & \vdots & & \vdots \\
	T_{(2^{N}-1)0} & T_{(2^{N}-1)1} & \ldots &
	T_{(2^{N}-1)(2^{N}-1)}
\end{array}\right) \] 
\cite{griffiths95quantum}.

Quantum mechanics imposes conditions on which linear transformations
are legal operators.  In particular, the operation must be reversible,
and it must preserve the length of the state vector
\cite{griffiths95quantum}.  If we impose the condition that the sum of
the kinetic and potential energy (called the Hamiltonian) of our
quantum memory register is constant, then all legal operators have
\emph{unitary} matrix representations.  A matrix $T$ is unitary if the
transpose of its complex conjugate is $T^{-1}$
\cite{griffiths95quantum}.  Systems with time-dependent Hamiltonians 
are not required to perform either Grover's or Shor's algorithm, and
are not within the scope of this thesis.

\subsubsection{Quantum Algorithms}
An operator taking a register in a superposition of states as an input
performs a ``computation'' on each of the components of the
superposition simultaneously.  Since the number of possible superposed
states is $2^{N}$ for an $N$-qubit register, in some sense a quantum
computer can perform in one operation what would take an exponential
number of operations on a classical computer.  Unfortunately, the more
superposed states, the smaller the probability that we will measure
the value of our function for any particular one.  Some clever
algorithms, most notably by Peter Shor and L. K. Grover, succeed by
considering a function for which some property of all the inputs is
useful.

A quantum memory register is just a state machine whose state
transitions are defined by the operators we apply.  A reversible
deterministic state machine with $2^{N}$ states can be modeled by an
$N$-bit register and operators which are $2^{N} \times 2^{N}$
permutation matrices.  A classical probabilistic state machine can be
modeled in a similar manner, as an $N$-bit register with a normalized
state vector in $\Real^{2^{N}}$ and operators that are doubly
stochastic matrices.  The difference between the quantum computer and
the probabilistic computer is that the projection of the quantum state
vector onto any eigenstate is a complex number: it has both a
magnitude and a phase.  The classical probabilistic state vector's
projection onto its eigenstates has only a positive real magnitude.
The amplitude of the quantum state vector allows for operators which
cause wave-like interference of the eigenstates, enforcing correct
solutions and diminishing incorrect ones
\cite{grover00search}.  

A quantum algorithm is described by an initial normalized state and a
sequence of unitary matrices representing linear transformations of
the state vector.  These unitary matrices should increase the
probability that the state vector collapses into a correct solution
state when measured.

\subsection{Error Models}
\label{sec:error}

Quantum computation is probabilistic in nature.  We must therefore
consider what (if any) error we will accept from our quantum
algorithms.  Different settings allow different expected running
times, just as in the classical case, where different running times
can be found for different types of randomized algorithms.

Three primary settings for quantum algorithms are defined by Beals et
al.\ \cite{beals98quantum}:
\begin{enumerate}

\item \textbf{The Exact Setting:} 
The quantum algorithm is required to return $f(x)$ with certainty for
every $x \in \{0,1\}^{N}$.

\item \textbf{The Zero Error Setting:}
The quantum algorithm may be inconclusive with probability at most
$1/2$, but if it returns an answer, it must be correct.

\item \textbf{The Bounded Error Setting:}
The quantum algorithm is required to return $f(x)$ with probability at
least $2/3$ for every $x \in \{0,1\}^{N}$ .
\end{enumerate}

Clearly any algorithm in the exact setting is also correct in the
bounded error setting.  Algorithms in the zero error setting can be
placed in the bounded error setting by performing multiple runs that
return an arbitrary value whenever they would return ``inconclusive.''
Thus the bounded error setting is the broadest category, and as such
offers the best potential speedups.  All the results in this thesis
refer to computations in the bounded error setting.

\section{Lower Bounds}
\label{sec:lower}

Given that we can theoretically attain exponential speedup through
quantum parallelism, it is tempting to hope that quantum computing
could offer exponential speedup to large classes of problems.  Shor's
algorithm seems to validate that hope.  The optimality of Grover's
algorithm, on the other hand, essentially establishes that a quantum
computer can offer at most quadratic speedup to the brute force
solution of a problem that exhaustively searches all possible
solutions.

To answer questions surrounding the power of the quantum computer, we
study lower bounds for quantum algorithms.  Lower bounds on large
classes of functions allow us to make quantitative comparisons between
quantum and classical computers.  It is very difficult to prove a
lower bound on a particular problem in any general model of
computation, as one has to argue how quickly any possible algorithm
can solve the problem.  Throughout this thesis we will consider a
restricted model of computation, the \emph{oracle query model}, in
which we consider only algorithms whose input is provided in a black
box.

We will present a theorem by Andris Ambainis that enables us to easily
prove lower bounds in the oracle query model
\cite{ambainis00quantum}.  The main strength of Ambainis' Theorem
is its ability to prove lower bounds for a variety of Boolean
functions, and thus allow us to compare the relative power of
classical and quantum computers.

\section{Summary}
\label{sec:outline}

In Chapter \ref{ch:oracle}, we define the quantum oracle model, and
present Ambainis' Theorem for proving lower bounds within this
framework.  We examine a shortcoming of Ambainis' Theorem, and then
derive two theorems and use them to prove lower query bounds for
symmetric functions.

In Chapter \ref{ch:graph}, we apply the results of Chapter
\ref{ch:oracle} to the interesting and well studied case of
non-trivial monotone graph properties.  Finding the results wanting we
appeal directly to Ambainis' Theorem to establish new lower bounds for
graph connectivity and bipartiteness.  Finally we show there is no
quantum extension of the Aanderaa-Karp-Rosenberg conjecture in the
bounded error setting.

In Chapter \ref{ch:general}, we examine classes of functions that do
not have the inherent symmetries of those in chapters \ref{ch:oracle}
and \ref{ch:graph}.  We provide lower query bounds for tree functions,
nondeterministically evasive functions, and sensitive functions.

We conclude in Chapter \ref{ch:open} with a brief examination of some
open questions for lower oracle query bounds in the bounded error
setting, and quantum computing in general.

\label{sec:sumofres}

Many of the results in this thesis have been attained previously by
other authors. We present these results again to illustrate how
Ambainis' Theorem can provide relatively simpler proofs for a variety
of problems.  Table \ref{tbl:results} summarizes our results and
credits previous papers where appropriate.  All results are in the
quantum bounded error setting.

\begin{table}
\begin{center}
\begin{tabular}{|l|l|l|l|}
\hline
Function & Lower Query Bound & Section & Reference \\ 
\hline
Generalized XOR & $\Omega\left(\sqrt{N}\right)$ & \ref{sec:gXOR} &  \\
\hline
Determining the Oracle String & $N/2$ & \ref{sec:dos} &  \\
\hline
Singleton functions & $\Omega\left(\sqrt{N}\right)$ & \ref{sec:torf1} & \\
\hline
Partially symmetric & $\Omega\left(\sqrt{\frac{(N-a)b}{(b-a)^{2}}}\right)$ & \ref{sec:lqbsym} &  \\
\hline
AND & $\Omega\left(\sqrt{N}\right)$ & \ref{sec:AOMP} & Beals et al.\ \cite{beals98quantum} \\
\hline
OR & $\Omega\left(\sqrt{N}\right)$ & \ref{sec:AOMP} & Beals et al.\ \cite{beals98quantum} \\
\hline
MAJORITY & $\Omega\left(N\right)$ & \ref{sec:AOMP} & Beals et al.\ \cite{beals98quantum} \\
\hline
PARITY & $\Omega\left(N\right)$ & \ref{sec:AOMP} & Beals et al.\ \cite{beals98quantum} \\
\hline
Nonconstant symmetric & $\Omega\left(\sqrt{N}\right)$ & \ref{sec:lqbnsf} & Beals et al.\ \cite{beals98quantum} \\
\hline
Graph connectivity & $\Omega\left(V\right)$ & \ref{sec:gc} & \\
\hline
Graph bipartiteness & $\Omega\left(V\right)$ & \ref{sec:bip} & \\
\hline
Tree functions & $\Omega\left(\sqrt[4]{N}\right)$ & \ref{sec:tree} & Beals et al.\ \cite{beals98quantum} \\
\hline
Nondeterministically evasive & $\Omega\left(\sqrt{N}\right)$ & \ref{sec:b4r} &  \\
\hline
Sensitive functions & $\Omega\left(\sqrt[4]{D(f)}\right)$ & \ref{sec:lqbcbf} & \\
\hline
\end{tabular}
\end{center}
\caption{Summary of Results \label{tbl:results}}
\end{table}

\include{2-related}
\include{3-model}
\include{4-predictions}
% \end{verbatim} \hrule \end{quote}
% \iffalse
%<example>
% \fi
%
% \subsection{Reference Matter}
%
% \DescribeMacro{\appendix}
% To switch from the body of your thesis to the reference material
% at the end, you should use the standard \LaTeX\ |\appendix| command.
%
% \begin{quote} \hrule \begin{verbatim}
\appendix

\include{A-data.tex}
% \end{verbatim} \hrule \end{quote}
% \iffalse
%<example>
% \fi
%
% \subsection{Back Matter}
%
% \DescribeMacro{\backmatter}
% The last few chapters in your thesis should not have chapter
% numbers, but should be listed in the Table of Contents.
% These chapters include the Bibliography, the Index,
% and the Vita.  \LaTeX's |\backmatter| command provides the correct
% formatting for these chapters.
%
% \DescribeMacro{\bibliography}
% Use the standard \LaTeX\ bibliography commands to
% create your bibliography---\pkg{uiucthesis} redefines them so
% that they produce the correct formatting for a UIUC thesis.
%
% \DescribeMacro{\vita}
% The |\vita| command begins a new chapter for your vita.
% In fact, it does exactly the same thing as |\chapter{\vitaname}|,
% where |\vitaname| is ``Vita.''
%
% \begin{quote} \hrule \begin{verbatim}
\backmatter

\bibliography{thesisbib}

\chapter{Vita}

Juan Valdez was born\ldots.

\end{document}
% \end{verbatim} \hrule \end{quote}
%
% \iffalse
%</example>
% \fi
%
% \StopEventually{%
%   \begin{thebibliography}{9}
%     \bibitem{Instructions94}
%     \emph{Instructions for Preparation of Theses: How to Deposit a Thesis
%     with the Graduate College and Keep Your Sanity}.
%     \newblock Graduate College, University of Illinois at
%     Urbana-Champaign, 1994
%
%     \bibitem{Lamport86}
%     Leslie Lamport.
%     \newblock \emph{\LaTeX: A Document Preparation System}.
%     \newblock Addison-Wesley, 1986.
%   \end{thebibliography}
% }
%
% \section{Implementation}
%
% This section shows the implementation of the \pkg{uiucthesis} package.
% Unless you are interested in the details of how \pkg{uiucthesis} works,
% you probably don't need to read it.
%
%    \begin{macrocode}
%<*package>
%    \end{macrocode}
%
% \iffalse
%<package>\usepackage{setspace}
% \fi
%
% \subsection{Compatibility}
%
% Provide compatibililty with older versions of LaTeX.
% \begin{macro}{\@ifundefined}
%    \begin{macrocode}
\expandafter\ifx\csname @ifundefined\endcsname\relax
  \def\@ifundefined#1{%
    \expandafter\ifx\csname#1\endcsname\relax
      \expandafter\@firstoftwo
    \else
      \expandafter\@secondoftwo
    \fi}
\fi
%    \end{macrocode}
% \end{macro}
%
% \begin{macro}{\MakeUppercase}
%    \begin{macrocode}
\@ifundefined{MakeUppercase}{\let\MakeUppercase=\uppercase}{}
%    \end{macrocode}
% \end{macro}
%
% \subsection{Option Processing}
%
%    \begin{macrocode}
\newif\if@thesisdraft \@thesisdraftfalse
\newif\if@thesisfancy \@thesisfancyfalse
\newif\if@fullpage \@fullpagefalse
\newif\if@largecaps \@largecapsfalse
%    \end{macrocode}
%
%    \begin{macrocode}
\DeclareOption{draftthesis}{%
  \@thesisdrafttrue
  \newcount\timehh\newcount\timemm
  \timehh=\time \divide\timehh by 60
  \timemm=\time \count255=\timehh \multiply\count255 by -60
    \advance\timemm by \count255
  \def\draftheader{Draft of \today\ at
  \ifnum\timehh<10 0\fi\number\timehh\,:\,\ifnum\timemm<10 0\fi\number\timemm}%
}
%    \end{macrocode}
%
%    \begin{macrocode}
\DeclareOption{fancy}{\@thesisfancytrue}
%    \end{macrocode}
%
%    \begin{macrocode}
\DeclareOption{fullpage}{\@fullpagetrue}
%    \end{macrocode}
%
% The |[largecaps]| option causes the title and author's name to
% be use a ``large caps'' font on the title page.  Otherwise,
% \pkg{uiucthesis} just converts them to uppercase and uses the
% normal fonts.  The difference is that the spacing between the
% characters in the large caps font is tuned for setting type in all caps.
%
% The large caps font is \emph{not a standard font}, and so it will not exist
% unless you have installed it.
%
%    \begin{macrocode}
\DeclareOption{largecaps}{\@largecapstrue}
%    \end{macrocode}
%
%    \begin{macrocode}
\ProcessOptions
%    \end{macrocode}
%
%
% \subsection{Title Page}
%
% \begin{macro}{\title}
% \begin{macro}{\author}
% Override the standard definitions of |\title| and |\author| to also
% define uppercased versions.
%    \begin{macrocode}
\def\@mkuptitle#1{\gdef\@Utitle{#1}}
\def\title#1{\gdef\@title{#1}\MakeUppercase{\protect\@mkuptitle{#1}}}
\def\@mkupauthor#1{\gdef\@Uauthor{#1}}
\def\author#1{\gdef\@author{#1}\MakeUppercase{\protect\@mkupauthor{#1}}}
%    \end{macrocode}
% \end{macro}
% \end{macro}
%
% \begin{macro}{\phdthesis}
% \begin{macro}{\msthesis}
% \begin{macro}{\department}
% \begin{macro}{\college}
% \begin{macro}{\schools}
% \begin{macro}{\degreeyear}
% Macros to set title page elements.
%    \begin{macrocode}
\def\phdthesis{\def\@degree{Doctor of Philosophy}}
\def\msthesis{\def\@degree{Master of Science}}
\def\department#1{\def\@dept{#1}}
\def\college#1{\def\@college{#1}}
\def\schools#1{\def\@schools{#1}}
\def\degreeyear#1{\def\@degreeyear{#1}}
%    \end{macrocode}
% \end{macro}
% \end{macro}
% \end{macro}
% \end{macro}
% \end{macro}
% \end{macro}
%
% \begin{macro}{\copyrightnotice}
% Define the copyright notice as a macro so that the user
% can change it if desired.
%    \begin{macrocode}
\def\copyrightnotice{\copyright~Copyright by \@author, \@degreeyear}
%    \end{macrocode}
% \end{macro}
%
% Set the default title page elements.
%    \begin{macrocode}
\phdthesis
\department{Computer Science}
\college{Graduate College}
\def\@schools{}
\def\@degreeyear{\number\year}
%    \end{macrocode}
%
%
% \begin{macro}{\maketitle}
% Redefine \pkg{report}'s |\maketitle| command to produce
% the titlepage in the correct format.
%
%    \begin{macrocode}
\renewcommand\maketitle{
%    \end{macrocode}
% If this is not a draft version, print the copyright page.
%    \begin{macrocode}
  \if@thesisdraft\else
    \newpage
    \thispagestyle{empty}
    \setcounter{page}{0}	% Copyright page is page 0
    \null\vfill
    \centerline{\copyrightnotice}%
    \vskip 36pt			% skip to visually center copyright notice
    \vfill
  \fi
%    \end{macrocode}
% Now start a new page for the title page.  It is single-spaced.
%    \begin{macrocode}
  \newpage
  \thispagestyle{empty}%
  \begingroup
  \def\baselinestretch{1}\@normalsize
%    \end{macrocode}
% We have to be careful to get the vertical position right.  The
% easiest way to do this seems to be to just set |\topmargin|,
% |\headheight|, and |\headsep| for this page.
%    \begin{macrocode}
  \topmargin=1.25in \headheight=0pt \headsep=0pt
%    \end{macrocode}
% Adjust the horizontal spacing so that the title page
% is centered on the page even if the rest of the document isn't.  
% I'm not sure when |\textwidth| changes take place, so instead 
% we calculate the correct |\oddsidemargin| to center the text column.
%    \begin{macrocode}
  \@tempdima=\paperwidth
  \advance\@tempdima by -\textwidth
  \divide\@tempdima by 2
  \advance\@tempdima by -1in
  \oddsidemargin=\@tempdima
  \let\evensidemargin=\oddsidemargin
  \if@largecaps
    \def\lc@selectfont{\fontshape{lc}\selectfont}%
  \else
    \def\lc@selectfont{}%
  \fi
  \begin{center}
  \vbox to 1.75in{%
    {\large\lc@selectfont\@Utitle}
    \if@thesisdraft
      \\[12pt]
      Draft Version\\
      \today
    \fi
    \vfil}%
  \vbox to 2.5in{%
    {\lc@selectfont BY}\\[12pt]
    {\large\lc@selectfont\@Uauthor}\\[12pt]
    \begin{tabular}{c}\@schools\end{tabular}\vfil}%
  \vbox to 2.5in{%
    {\lc@selectfont THESIS}\\[12pt]
    Submitted in partial fulfillment of the requirements\\
    for the degree of \@degree\ in \@dept\\
    in the \@college\ of the\\
    University of Illinois at Urbana-Champaign, \@degreeyear\vfil}
  Urbana, Illinois
  \end{center}
  \newpage
  \endgroup
%    \end{macrocode}
% Skip Page ii for Certificate of Committee Approval.
%    \begin{macrocode}
%  \newpage
%  \setcounter{page}{3}
}
%    \end{macrocode}
% \end{macro}
%
%
% \section{Front Matter}
%
% \begin{macro}{\frontmatter}
% Redefine |\frontmatter| so that it sets the page number to 3.
%    \begin{macrocode}
\let\thesis@frontmatter=\frontmatter
\def\frontmatter{%
  \thesis@frontmatter
  \setcounter{page}{3}}
%    \end{macrocode}
% \end{macro}
%
% \subsection{Table of Contents}
%
% \begin{macro}{\contentsname}
% Use ``Table of Contents'' instead of ``Contents.''
% (Is this actually a format requirement?)
%    \begin{macrocode}
\renewcommand\contentsname{Table of Contents}
%    \end{macrocode}
% \end{macro}
%
% \begin{macro}{\l@chapter}
%    This code is a modified version of the code in the 1996/05/26 release
%    of classes.dtx that produces leader dots between the chapter
%    name and the page number.
%
%    This macro formats the entries in the table of contents for
%    chapters. It is very similar to |\l@part|
%
%    First we make sure that if a pagebreak should occur, it occurs
%    \emph{before} this entry. Also a little whitespace is added and a
%    group begun to keep changes local.
% \changes{v1.2i}{1994/04/28}{Don't print a toc line when the tocdepth
%    counter is less then 0}
% \changes{v1.3b}{1955/05/23}{Added missing braces around argument
%           to \cs{addpenalty}.}
%    \begin{macrocode}
\renewcommand*\l@chapter[2]{%
  \ifnum \c@tocdepth >\m@ne
    \addpenalty{-\@highpenalty}%
    \vskip 1.0em \@plus\p@
%    \end{macrocode}
%
%    The macro |\numberline| requires that the width of the box that
%    holds the part number is stored in \LaTeX's scratch register
%    |\@tempdima|. Therefore we put it there. We begin a group, and
%    change some of the paragraph parameters.  These are different
%    from the defaults for the standard report or book class.
%    \begin{macrocode}
    \setlength\@tempdima{1.5em}
    \begingroup
      \leftskip \z@ \rightskip \@tocrmarg \parfillskip -\rightskip
      \parindent \z@
%    \end{macrocode}
%    Then we leave vertical mode and switch to a bold font.
%    \begin{macrocode}
      \leavevmode \bfseries
%    \end{macrocode}
%    Because we do not use |\numberline| here, we have do some fine
%    tuning `by hand', before we can set the entry. We discourage but
%    not disallow a pagebreak immediately after a chapter entry.
%    We use leaders between the chapter title and the page number,
%    unlike the standard report or book class.
%    \begin{macrocode}
      \advance\leftskip\@tempdima
      \hskip -\leftskip
      #1\nobreak
      \leaders\hbox{$\m@th\mkern\@dotsep mu\hbox{.}\mkern\@dotsep mu$}
      \hfil \nobreak\hbox to\@pnumwidth{\hss #2}\par
      \penalty\@highpenalty
    \endgroup
  \fi}
%    \end{macrocode}
% \end{macro}
%
% \begin{macro}{\tableofcontents}
% We want the Table of Contents to be single-spaced, so
% we save the original definition, and then arrange it so
% that the new |\tableofcontents| calls |\singlespacing| before calling
% the original definition.
%    \begin{macrocode}
\let\thesis@tableofcontents=\tableofcontents
\def\tableofcontents{{\singlespacing\thesis@tableofcontents}}
%    \end{macrocode}
% \end{macro}
%
% \begin{macro}{\listoftables}
% \begin{macro}{\listoffigures}
% Similarly, redefine |\listoftables| and |\listoffigures| so
% that they use single spacing.  Why bother, you ask?
% Well, it {\em is} possible that some user might
% call |\listoffigures| without calling |\tableofcontents|.  Right?
%    \begin{macrocode}
\let\thesis@listoftables=\listoftables
\def\listoftables{{\singlespacing\thesis@listoftables\thesisspacing}}
\let\thesis@listoffigures=\listoffigures
\def\listoffigures{{\singlespacing\thesis@listoffigures}}
%    \end{macrocode}
% \end{macro}
% \end{macro}
%
%
% \subsection{Chapter Headings}
%
% Text of chapter title must match exactly with text in Table of Contents.
% We support both plain chapter headings and ``fancy'' chapter headings.
%
% \begin{macro}{\chapternumberfont}
%    Define the font used for chapter numbers in fancy chapter headings.
%    If you're using scalable PostScript fonts, you might want to
%    override it, for example:
%    \begin{verbatim}
%    \renewcommand\chapternumberfont{
%       \fontseries{bx}\fontsize{72}{72}\selectfont}
%    \end{verbatim}
% \begin{macrocode}
\if@thesisfancy
  \font\cminch=cminch at 60pt
  \newcommand\chapternumberfont{\cminch}
\else
  \newcommand\chapternumberfont{\huge\bfseries}
\fi
% \end{macrocode}
%    
% \begin{macro}{\chaptertitlefont}
%    Define the font used for chapter titles.
% \begin{macrocode}
\newcommand\chaptertitlefont{\Huge\bfseries}
% \end{macrocode}
%    
% \begin{macro}{\@chapter}
%    This macro is called when we have a numbered chapter. When
%    |secnumdepth| is larger than $-1$ and, in the book
%    class, |\@mainmatter| is true, we display the chapter
%    number. We also inform the user that a new chapter is about to be
%    typeset by writing a message to the terminal.
%    \begin{macrocode}
\def\@chapter[#1]#2{%
  \ifnum \c@secnumdepth >\m@ne
    \if@mainmatter
      \refstepcounter{chapter}%
      \typeout{\@chapapp\space\thechapter.}%
      \if@thesisfancy
        \addcontentsline{toc}{chapter}%
          {\protect\numberline{\thechapter}#1}%
      \else
        \addcontentsline{toc}{chapter}%
          {\@chapapp\ \thechapter\quad #1}%
      \fi
    \else
      \addcontentsline{toc}{chapter}{#1}%
    \fi
  \else
    \addcontentsline{toc}{chapter}{#1}%
  \fi
%    \end{macrocode}
%    After having written an entry to the table of contents we store
%    the (alternative) title of this chapter with |\chaptermark| and
%    add some white space to the lists of figures and tables.
%    \begin{macrocode}
  \chaptermark{#1}%
  \addtocontents{lof}{\protect\addvspace{10\p@}}%
  \addtocontents{lot}{\protect\addvspace{10\p@}}%
%    \end{macrocode}
%    Then we call upon |\@makechapterhead| to format the actual
%    chapter title. We have to do this in a special way when we are in
%    twocolumn mode in order to have the chapter title use the entire
%    |\textwidth|. In one column mode we call |\@afterheading|, which
%    takes care of suppressing the indentation.
%    \begin{macrocode}
  \if@twocolumn
    \@topnewpage[\@makechapterhead{#2}]%
  \else
    \@makechapterhead{#2}%
    \@afterheading
  \fi}
%    \end{macrocode}
% \end{macro}
% 
% For fancy chapter headings, compute the correct height to use for the
% chapter number.  I want the chapter number to be centered on the first line
% of the chapter title.  If $a$ is the height of the chapter number and $b$ is
% the height of the chapter title, then if we set the chapter number in a
% box of height $b + (a-b)/2\quad=\quad(a+b)/2$ then it aligns correctly.
%
% We arrange for this value to be computed at the beginning of the document
% in case the user loads a style file that changed the default fonts.
%    \begin{macrocode}
\if@thesisfancy
  \AtBeginDocument{%
    \newdimen\chapternumberheight
    {\setbox255=\hbox{\chapternumberfont A}
     \dimen255=\ht255
     \setbox255=\hbox{\Huge\bfseries A}
     \advance\dimen255 by \ht255
     \divide\dimen255 by 2
     \global\chapternumberheight=\dimen255}
  }
\fi
%    \end{macrocode}
%
% \begin{macro}{\@makechapterhead}
%    The macro |\@chapter| uses |\@makechapterhead|\meta{text} to format the
%    heading of the chapter.  This is a modified version of the standard
%    |\@makechapterhead|.  It sets the chapter heading in single spacing,
%    and it handles the fancy heading style, 
%
%    We begin by leaving some white space. The we open a group in
%    which we have a paragraph indent of 0pt, and in which we have the
%    text set ragged right. We also reset the font.
%    \begin{macrocode}
\def\@makechapterhead#1{%
  \if@thesisfancy\vspace*{10\p@}\else\vspace*{50\p@}\fi
  {\def\baselinestretch{1}\@normalsize
    \parindent \z@ \raggedright \normalfont
%    \end{macrocode}
%    Then we check whether the number of the chapter has to be printed.
%    If so we leave some whitespace between the chapternumber and its
%    title.
%    \begin{macrocode}
    \ifnum \c@secnumdepth >\m@ne
      \if@mainmatter
        \if@thesisfancy
          \leavevmode\llap{\vbox to \chapternumberheight{\hbox{%
            \chapternumberfont\thechapter\ }\vss}}%
        \else
          \chapternumberfont \@chapapp\space \thechapter
          \par\nobreak
          \vskip 20\p@
        \fi
      \fi
    \fi
%    \end{macrocode}
%    Now we set the title in a large bold font. We prevent a pagebreak
%    from occuring in the middle of or after the title. Finally we
%    leave some whitespace before the text begins.
%    \begin{macrocode}
    \interlinepenalty\@M
    \chaptertitlefont #1\par\nobreak
    \vskip 40\p@
  }}
%    \end{macrocode}
% \end{macro}
%
% \begin{macro}{\@makeschapterhead}
%    The macro |\@schapter| uses |\@makeschapterhead|\meta{text}to format
%    the heading of the chapter. It is similar to |\@makechapterhead|
%    except that it never has to print a chapter number.
%
%    \begin{macrocode}
\def\@makeschapterhead#1{%
  \if@thesisfancy\vspace*{10\p@}\else\vspace*{50\p@}\fi
  %\vspace*{50\p@}%
  {\def\baselinestretch{1}\@normalsize
    \parindent \z@ \raggedright
    \normalfont
    \interlinepenalty\@M
    \Huge \bfseries  #1\par\nobreak
    \vskip 40\p@
  }}
%    \end{macrocode}
% \end{macro}
%
% \subsection{Lower Level Headings}
%
% We redefine the lower level headings to set their titles
% ragged right.  These are modified from \pkg{classes.dtx}.
% We don't have to change sectioning commands below
% \cs{subsubsection} because they produce run-in headings.
%
% \begin{macro}{\section}
%    \begin{macrocode}
\renewcommand\section{\@startsection {section}{1}{\z@}%
  {-3.5ex \@plus -1ex \@minus -.2ex}%
  {2.3ex \@plus.2ex}%
  {\raggedright\normalfont\Large\bfseries}}
%    \end{macrocode}
% \end{macro}
%
% \begin{macro}{\subsection}
%    \begin{macrocode}
\renewcommand\subsection{\@startsection{subsection}{2}{\z@}%
  {-3.25ex\@plus -1ex \@minus -.2ex}%
  {1.5ex \@plus .2ex}%
  {\raggedright\normalfont\large\bfseries}}
%    \end{macrocode}
% \end{macro}
%
% \begin{macro}{\subsubsection}
%    \begin{macrocode}
\renewcommand\subsubsection{\@startsection{subsubsection}{3}{\z@}%
  {-3.25ex\@plus -1ex \@minus -.2ex}%
  {1.5ex \@plus .2ex}%
  {\raggedright\normalfont\normalsize\bfseries}}
%    \end{macrocode}
% \end{macro}
%
% \subsection{Appendix}
%
% ``Instructions for the Preparation of Theses,'' 1994, says,
% ``Do not number appendices as chapters[.]''  However, I checked
% with the Thesis Office (16 July 1996) and was told that the
% LaTeX style of lettering appendices (A, B, C, etc.) was OK.
% However, neither the Bibliography nor Vita are considered
% appendices, so they should appear in the Table of Contents,
% but they should not be given chapter numbers or appendix letters.
% So: no changes to the standard \LaTeX\ style are necessary.
%
%
% \subsection{Bibliography}
%
% \begin{macro}{\bibname}
% UIUC Thesis format says that if references are cited as ``[1]'' then
% one of the terms ``References,'' ``List of References,'' or 
% ``Literature Cited'' should be used instead of ``Bibliography.''
%    \begin{macrocode}
\renewcommand\bibname{References}
%    \end{macrocode}
% \end{macro}
%
% \begin{environment}{thebibliography}
%    This |thebibliography| environment is the same as the standard
%    Except it calls |\chapter| instead of |\chapter*|.
%
%    The `thebibliography' environment executes the following
%    commands:
%
%    |\renewcommand{\newblock}{\hskip.11em \@plus.33em \@minus.07em}|\\
%      --- Defines the ``closed'' format, where the blocks (major units
%      of information) of an entry run together.
%
%    |\sloppy|  --- Used because it's rather hard to do line breaks in
%      bibliographies,
%
%    |\sfcode`\.=1000\relax| ---
%      Causes a `.' (period) not to produce an end-of-sentence space.
%
%    The implementation of this environment is based on the generic
%    list environment. It uses the |enumiv| counter internally
%    to generate the labels of the list.
%
%    When an empty `thebibliography' environment is found, a warning
%    is issued.
%
%    \begin{macrocode}
\renewenvironment{thebibliography}[1]
     {\chapter{\bibname
        \@mkboth{\MakeUppercase\bibname}{\MakeUppercase\bibname}}%
      \list{\@biblabel{\@arabic\c@enumiv}}%
           {\settowidth\labelwidth{\@biblabel{#1}}%
            \leftmargin\labelwidth
            \advance\leftmargin\labelsep
	    \@ifundefined{@openbib@code}{}{\@openbib@code}
            \usecounter{enumiv}%
            \let\p@enumiv\@empty
            \renewcommand\theenumiv{\@arabic\c@enumiv}}%
      \sloppy\clubpenalty4000\widowpenalty4000%
      \sfcode`\.\@m}
     {\def\@noitemerr
       {\@latex@warning{Empty `thebibliography' environment}}%
      \endlist}
%    \end{macrocode}
% \end{environment}
%
% \subsection{Index}
%
% \begin{environment}{theindex}
% The index is single spaced.
%    \begin{macrocode}
\let\thesis@theindex=\theindex
\def\theindex{\begingroup\singlespacing\thesis@theindex}
\let\thesis@endtheindex=\endtheindex
\def\endtheindex{\thesis@endtheindex\endgroup}
%    \end{macrocode}
% \end{environment}
%
%
% \subsection{Page Layout}
%
% First we set the vertical layout.
% Adjust the height of the text column so that it takes up the full
% height of an 8.5 by 11 inch page.
%    \begin{macrocode}
\topmargin=.5in
\advance \topmargin by -\headheight
\advance \topmargin by -\headsep
\textheight 8.9in
%    \end{macrocode}
% Next, set the horizontal layout.
%
% The standard for bad typography in technical papers seems to be to use
% extremely wide columns of text, and then to increase the spacing between
% lines to compensate for the long lines.
% Unfortunately, because so many papers are typeset this way,
% the format has become self-propagating.
%
% I concede defeat.  The |[fullpage]| option sets one-inch margins.
%    \begin{macrocode}
\if@fullpage
  \oddsidemargin 0pt
  \evensidemargin \oddsidemargin
  \marginparwidth 0.5in
  \textwidth\paperwidth
  \advance\textwidth by -2in
\fi
%    \end{macrocode}
% The ``fancy'' thesis style uses an asymmetrical page layout,
% shifting the text column slightly over to the right to leave
% room for the chapter number to the left of the chapter title.
%    \begin{macrocode}
\if@thesisfancy
  \advance\oddsidemargin by 24pt
  \let\evensidemargin=\oddsidemargin
%    \end{macrocode}
% Set |\marginparwidth|, leaving 36pt for the right margin.
% Note that you're not allowed to actually use a marginal paragraph 
% this close to the edge in the final version of a thesis, but it is still
% handy for leaving notes to yourself in the draft.
%    \begin{macrocode}
  \marginparwidth=\oddsidemargin
  \advance\marginparwidth by 1in
  \advance\marginparwidth by -\marginparsep
  \advance\marginparwidth by -36pt
%    \end{macrocode}
% Put the marginal paragraphs on the left side, where there's more room.
%    \begin{macrocode}
  \reversemarginpar
\fi
%    \end{macrocode}
%    
% \subsection{Page Numbering}
%
% Page numbers must be in one of three places, and must appear in the
% same place on \textit{every page}, including chapter openings.
%
% To accomodate the draft heading, we redefine the plain page style.
% \begin{macro}{\ps@plain}
%    \begin{macrocode}
\def\ps@plain{%
  \let\@mkboth\@gobbletwo
  \if@thesisdraft
    \def\@oddhead{\draftheader\hfil}
  \else
    \let\@oddhead\@empty
  \fi
  \let\@evenhead\@oddhead
  \def\@oddfoot{\reset@font\hfil\thepage\hfil}%
  \let\@evenfoot\@oddfoot
}
%    \end{macrocode}
% \end{macro}
%
% Then we set the default page style to (our new definition of) plain.
%    \begin{macrocode}
\pagestyle{plain}
%    \end{macrocode}
%
% \begin{macro}{\chapter}
% Redefine |\chapter| to not do |\thispagestyle{empty}| because
% even chapter openings should have page numbers in UIUC Thesis style.
%    \begin{macrocode}
\renewcommand\chapter{\if@openright\cleardoublepage\else\clearpage\fi
  \global\@topnum\z@
  \@afterindentfalse
  \secdef\@chapter\@schapter}
%    \end{macrocode}
% \end{macro}
%
%
% \subsection{Vita}
%
% The support for |\vita| is pretty minimal.
%
% \begin{macro}{\vitaname}
% Define |\vitaname| so the user can change it if he wants.
%    \begin{macrocode}
\newcommand\vitaname{Vita}
%    \end{macrocode}
% \end{macro}
%
% \begin{macro}{\vita}
% Vita should appear in Table of Contents, but should not be numbered.
%    \begin{macrocode}
\newcommand\vita{
  \chapter{\vitaname}%
}
%    \end{macrocode}
% \end{macro}
%
% \subsection{Body Formatting}
%
% \begin{macro}{\thesisspacing}
% The |\thesisspacing| command is called to switch to the default
% line spacing for the thesis.  The thesis format requirements
% require at least line and a half spacing.  
% The \pkg{uiucthesis} package by default uses |\onehalfspacing|,
% or |\doublespacing| if the |[fullpage]| option is in effect.
%    \begin{macrocode}
\def\thesisspacing{\if@fullpage\doublespacing\else\onehalfspacing\fi}
%    \end{macrocode}
% \end{macro}
%
% At this point, we're ready to set up the actual formatting for the
% front matter of the thesis.  We use roman page numbers.
% Also, arrange so that |\thesisspacing| gets called when the document
% begins.  We don't just call it here because that wouldn't give the
% user a chance to override it.
%    \begin{macrocode}
\pagenumbering{roman}
\AtBeginDocument{\thesisspacing}
%    \end{macrocode}
%
% \subsection{Compatibility}
%
% \begin{macro}{preliminary}
% The old \pkg{uiucthesis} style defined a \env{preliminary}
% environment for the front matter.  This isn't needed with this
% style, so we redefine it to call |\frontmatter| for compatibility's sake.
%    \begin{macrocode}
\def\preliminary{\frontmatter}
\let\endpreliminary=\relax
%    \end{macrocode}
% \end{macro}
%
% \begin{macro}{thesis}
% Similarly, the old \pkg{uiucthesis} style defined a \env{thesis}
% environment that has been superceded by the |\mainmatter| command.
% We define it here for backward compatibility.
%    \begin{macrocode}
\def\thesis{\mainmatter}
\let\endthesis=\relax
%    \end{macrocode}
% \end{macro}
%
%    \begin{macrocode}
%</package>
%    \end{macrocode}
% \Finale
\endinput
